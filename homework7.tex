\documentclass{article}
\usepackage[utf8]{inputenc}
\usepackage{amsmath, amsthm, amssymb, amsfonts}
\usepackage{tikz}
\usepackage{graphicx}
\graphicspath{ {./images/} }

\title{Homework 7}
\author{DUMA Yehor (31209504)}
\date{12/9}
\begin{document}

\maketitle

\section{Exercise 3}

Let $A$ be a set of all negative integers, $A = \{-1,-2,-3,...\}$. By definition, a set is infinite iff it is equivalent to a proper subset of itself. Consider set $B=\{-2,-3-4,...\}, B \subset A$. Suppose there is a mapping $M:A \xrightarrow[]{}B$, such that $M(x)=x-1$. To each member of $A$, then, there corresponds a a unique member of $B$. Therefore, $M$ is a one-to-one correspondence, and $A \sim B$. As $A \supset B$ and  $A \sim B$, $A$ is infinite. 
\section{Exercise 4}
Suppose we put all possible English sentences into a print dictionary, arranged by length and alphabetically. $S = \{a, b, c,...,aa,ab,...\}$. To each sentence we assign a unique natural number. Then all sentences can be printed on paper in linear order, with a one-to-one correspondence between the sentences and the natural numbers, hence $S\sim \mathbb{N}$ and $|S|=|\mathbb{N}|= \aleph{_0}$




\end{document}